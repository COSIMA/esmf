% $ Id: $
%
% Earth System Modeling Framework
% Copyright 2002-2009, University Corporation for Atmospheric Research, 
% Massachusetts Institute of Technology, Geophysical Fluid Dynamics 
% Laboratory, University of Michigan, National Centers for Environmental 
% Prediction, Los Alamos National Laboratory, Argonne National Laboratory, 
% NASA Goddard Space Flight Center.
% Licensed under the University of Illinois-NCSA License.

Each specific requirement possesses the following attributes:  priority, 
source, verification, status, and notes, the last of which is optional.  
These are typical for requirements
analysis ~\cite{wiegers}.  We'll now look at them in more detail.

\begin{description}
\item [Priority] The purpose of the priority attribute is to associate
each requirement with the milestones and longer term project goals that 
it satisfies.  Each requirement is assigned a number from 1-3, with
values defined as follows:
\begin{enumerate}

\item This capability is directly required for a milestone OR
Half or more of the JMC applications that could use the utility or class
in which this capability is embedded would require this
capability in order to maintain their existing functionality;
 
\item Less than half of the JMC applications that could use
the utility or class in which this capability is embedded 
would require this capability in order to maintain their existing
functionality.

\item This capability is desired in order to extend
the existing functionality of one or more JMC codes.

\end{enumerate}

If some capability merits additional explanation to describe 
its priority, those preparing requirements are encouraged 
to elaborate.
 
\item [Source] The source attribute traces each capability
to the applications to which it applies.  In addition to applications
particular people or organizations may be noted.   This attribute 
helps to identify those that can provide further 
information and who may also be potential testers and users.  It
prevents the inclusion of features that have little likelihood of
being used.

\item [Verification] The verification attribute specifies an objective
and quantitive strategy for assessing whether a requirement is
satisfied.  Typical values include {\it code inspection}, 
{\it unit test} and {\it system test}.
Some capabilities may require the preparation of special data sets.

\item [Status] Throughout the course of this project it will be 
useful for us to track what has been accomplished and to archive 
ideas for extensions and improvements.  The status attribute identifies
each capability as:
\begin{itemize}
\item {\bf proposed}; this indicates an item that has been accepted as useful, but
that is not scheduled for implementation.
\item {\bf approved-1}; this indicates an item approved for implementing as part
of the 1st code release at Milestone F
\item {\bf approved-2}; this indicates an item approved for implementing as part 
of the 2nd code release at Milestone G
\item {\bf implemented}
\item {\bf verified}
\item {\bf rejected}; this indicates an item that has been actively rejected by
the review team.
\end{itemize}
Whether the cabability exists in other packages or models
is also helpful to note.

\item [Notes] This is a catch-all for additional information such
as background, references, related design and implementation issues, 
risk factors, and so on.

\end{description}


