% $Id: Attribute_obj.tex,v 1.11.2.1 2010/02/05 19:53:27 svasquez Exp $
%
% Earth System Modeling Framework
% Copyright 2002-2010, University Corporation for Atmospheric Research,
% Massachusetts Institute of Technology, Geophysical Fluid Dynamics
% Laboratory, University of Michigan, National Centers for Environmental
% Prediction, Los Alamos National Laboratory, Argonne National Laboratory,
% NASA Goddard Space Flight Center.
% Licensed under the University of Illinois-NCSA License.

Each Attribute contains a name-value pair in which the value can be any of several numeric, character, and logical types.  The allowable ESMF Attribute value types include:

\begin{itemize}
\item {\tt ESMF\_TYPEKIND\_I4}
\item {\tt ESMF\_TYPEKIND\_I4} list
\item {\tt ESMF\_TYPEKIND\_I8}
\item {\tt ESMF\_TYPEKIND\_I8} list
\item {\tt ESMF\_TYPEKIND\_R4}
\item {\tt ESMF\_TYPEKIND\_R4} list
\item {\tt ESMF\_TYPEKIND\_R8}
\item {\tt ESMF\_TYPEKIND\_R8} list
\item {\tt ESMF\_TYPEKIND\_Logical}
\item {\tt ESMF\_TYPEKIND\_Logical} list
\item {\tt EMSF\_TYPEKIND\_Character}
\item {\tt EMSF\_TYPEKIND\_Character} list
\end{itemize}

The other members of the Attribute class can be seen in Figure \ref{fig:AttributeClassUML}  which shows a UML representation of the ESMF Attribute object.   For a more detailed view of how Attribute packages and hierarchies are formed, see Figures \ref{fig:AttributePackageUML} and \ref{fig:AttributeHierarchyUML}, respectively.

\begin{figure}[h]
\centering
\includegraphics[width=4in,height=6.5in]{AttributeClassUML}
\caption{The structure of the Attribute class}
\label{fig:AttributeClassUML}
\end{figure}
\clearpage

\begin{figure}[h]
\centering
\includegraphics[width=5.5in,height=6in]{AttributePackageUML}
\caption{The internal object organization for the representation of Attribute packages}
\label{fig:AttributePackageUML}
\end{figure}
\clearpage

\begin{figure}[h]
\centering
\includegraphics[width=5.5in,height=6in]{AttributeHierarchyUML}
\caption{The internal object organization for the representation of Attribute hierarchies}
\label{fig:AttributeHierarchyUML}
\end{figure}
\clearpage
