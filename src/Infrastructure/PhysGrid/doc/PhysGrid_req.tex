%$Id: PhysGrid_req.tex,v 1.27.4.1 2006/11/16 00:15:38 cdeluca Exp $

% Earth System Modeling Framework
% Copyright 2002-2008, University Corporation for Atmospheric Research,
% Massachusetts Institute of Technology, Geophysical Fluid Dynamics
% Laboratory, University of Michigan, National Centers for Environmental
% Prediction, Los Alamos National Laboratory, Argonne National Laboratory,
% NASA Goddard Space Flight Center.
% Licensed under the University of Illinois-NCSA License.

%===============================================================================
\req{Physical locations}
%-------------------------------------------------------------------------------

A mechanism shall be provided for describing physical locations in space in 1,
2, or 3 dimensions, including both specification of points and of ranges.
\begin{reqlist}
{\bf Priority:} 1. \\
{\bf Source:} Required by CAM-EUL, CLM, CCSM-CPL, POP, CICE, 
              CAM-FV, PSAS, MIT, WRF, GFDL. \\
{\bf Status:} Approved-1. \\
{\bf Verification:} System test.
\end{reqlist}

\sreq{Horizontal locations}

\ssreq{Horizontal coordinates}

Physical domains may use Cartesian, spherical, or cylindrical coordinate
systems in the horizontal directions.  Units for these coordinates are meters
(for Cartesian), degrees of latitude and longitude (for spherical), and meters
and degrees for the radius and angle in cylindrical coordinates.

\begin{reqlist}
{\bf Priority:} 1. \\
{\bf Source:} Required by CAM-EUL, CLM, CCSM-CPL, POP, CICE, NCEP-GSM, NCEP-SSI, CAM-FV, PSAS, MIT, GFDL \\
{\bf Status:} Approved-1. \\
{\bf Verification:} Code inspection.\\
{\bf Notes:}  This suggestion follows common practice, but is an explicit
exception to the requirement that MKS units are used by ESMF codes wherever
units must be assumed. 
\end{reqlist}

\ssreq{Horizontal locations may be points}

Horizontal locations may be specified as a pair of real values in the order
(X,Y) or (longitude, latitude).
\begin{reqlist}
{\bf Priority:} 1. \\
{\bf Source:} Required by POP, CICE, Regrid, NCEP-GSM, NCEP-SSI,
              PSAS, MIT, WRF, GFDL. \\
{\bf Status:} Approved-1. \\
{\bf Verification:} Unit test.
\end{reqlist}


\ssreq{Horizontal locations may be polygonal regions}

  Horizontal locations may be specified to be regions by providing the number
of vertices and the list of the vertex points.  Vertex points must be specified
either clockwise or counterclockwise around the region.  The vertex points
may be redundant.
\begin{reqlist}
{\bf Priority:} 1. \\
{\bf Source:} Required by POP, CICE, Regrid, PSAS, MIT, GFDL. \\
{\bf Status:} Approved-1. \\
{\bf Verification:} Unit test.\\
{\bf Notes:} Fundamental to allowing conservative interpolation, and for a
precise description of data locations.
\end{reqlist}
 
\ssreq{Horizontal regions may have central points}

  Both a central point and a region may be specified in describing a horizontal
location.  The points may provide a convenient nominal location, even when
a value actually pertains to a region.
\begin{reqlist}
{\bf Priority:} 1. \\
{\bf Source:} Required by POP, CICE, Regrid, PSAS, MIT, GFDL. \\
{\bf Status:} Approved-1. \\
{\bf Verification:} Unit test.\\
{\bf Notes:} Many models mix finite difference and finite volume concepts.
\end{reqlist}

\ssreq{Horizontal regions may be circular}

  A horizontal location may be specified by adding a nominal radius of
influence to the central point.  This may be the radius of a Gaussian
distribution of influence. The exact interpretation of this radius is
the responsibility of user-provided software.
\begin{reqlist}
{\bf Priority:} 2. \\
{\bf Source:} Required by PSAS, MIT, GFDL. \\
{\bf Status:} Approved-2. \\
{\bf Verification:} Unit test.\\
{\bf Notes:} This is necessary for describing certain observational data streams.
\end{reqlist}


\ssreq{Paths between grid locations may be specified} 

A method for determining the path connecting grid locations 
is required.  This path would be used to accurately compute
intersections for Regridding, lengths of cell sides, grid
cell areas and a variety of other grid metrics.  A linear 
approximation between points is a proper assumption for 
cartesian grids and for computing sides of latitude/longitude
or reduced grid cells.  A linear approximation is also 
adequate in many other cases and would be a logical default choice. 
The most accurate solution would permit users to pass a 
subroutine which provides analytic or highly-accurate discrete 
forms of the grid Jacobian (the matrix of partial derivatives of 
the physical coordinates with respect to logical coordinates).  An 
additional possibility might internally support analytic forms like 
great circles or higher-order approximations (eg quadratic
approximation to the cell side given a midpoint in addition
to the two endpoints).

\begin{reqlist}
{\bf Priority:} 1. \\
{\bf Source:} Regrid, required by codes that use conservative interpolation. \\
{\bf Status:} Approved-1. \\
{\bf Verification:} Unit test.\\
{\bf Notes:} Necessary for allowing conservative interpolation.
\end{reqlist}

\sreq{Vertical locations}

\ssreq{Vertical coordinates}

Physical domains may use a variety of vertical coordinates, including pressure,
height, density, isotherms, sigma, other terrain-following, or any other
vertically monotonic quantity.  In addition, a user-interpretable vertical
proxy (such as a satellite measurement channel) may be used.  Units of this
coordinate must be self-consistent.  (See the CF convention for a full
discussion of options for vertical coordinates at
http://www.cgd.ucar.edu/cms/eaton/netcdf/CF-20010629.htm)
\begin{reqlist}
{\bf Priority:} 1. \\
{\bf Source:} Required by CAM-EUL, POP, NCEP-GSM, NCEP-SSI, CAM-FV, PSAS, MIT, WRF, GFDL. \\
{\bf Status:} Approved-1. \\
{\bf Verification:} Code inspection.
\end{reqlist}

\ssreq{Vertical locations may be points}
\begin{reqlist}
{\bf Priority:} 1. \\
{\bf Source:} Required by CAM-EUL, POP, NCEP-SSI,
              CAM-FV, PSAS, MIT, WRF, GFDL. \\
{\bf Status:} Approved-1. \\
{\bf Verification:} Unit test.
\end{reqlist}

\ssreq{Vertical locations may be regions}

Vertical locations may be specified by providing the values of the top and
bottom bounding points.  Such regions have the same extent regardless of the
order in which the bounding points are specified.
\begin{reqlist}
{\bf Priority:} 1. \\
{\bf Source:} Required by POP, MIT, GFDL. \\
{\bf Status:} Approved-1. \\
{\bf Verification:} Unit test.
\end{reqlist}

\ssreq{Vertical regions have central points}

  Both a central point and a region may be specified in describing a vertical
location.  The points may provide a convenient nominal location, even when
a value actually pertains to a region.
\begin{reqlist}
{\bf Priority:} 1. \\
{\bf Source:} Required by POP, NCEP-GSM, PSAS, MIT, GFDL. \\
{\bf Status:} Approved-1. \\
{\bf Verification:} Unit test.\\
{\bf Notes:} Many models mix finite difference and finite volume concepts.
\end{reqlist}

\ssreq{Vertical locations may have region of influence}

  A vertical location may be specified by adding a nominal radius of
influence to the central point.  This may be the radius of a Gaussian
distribution of influence. The exact interpretation of this radius is
the responsibility of user-provided software.
\begin{reqlist}
{\bf Priority:} 3. \\
{\bf Source:} \\
{\bf Status:} Proposed. \\
{\bf Verification:} Unit test.\\
{\bf Notes:} This appears superfluous. It is more than enough to specify a range 
in 1D - Arlindo.
\end{reqlist}
 
\ssreq{Vertical locations may include lopped cells}

  Vertical locations may include a region whose bounds vary between the
horizontal corners of a region.
\begin{reqlist}
{\bf Priority:} 2. \\
{\bf Source:} Required by MIT, NSIPP, POP, GFDL/MOM. \\
{\bf Status:} Approved-1. \\
{\bf Verification:} Unit test.\\
{\bf Note:} MOVE TO LATER - CNH
\end{reqlist}

%===============================================================================
\req{Location streams}
%-------------------------------------------------------------------------------

Streams of locations are used to describe the physical (and potentially temporal)
locations associated with streams of data.  Streams of locations differ from
physical grids (see below) in that there are no concept of neighboring values,
topology, covering a space, or of locations being exclusive.
\begin{reqlist}
{\bf Priority:} 1. \\
{\bf Source:} Required by NCEP-SSI, PSAS, MIT, GFDL. \\
{\bf Status:} Approved-2. \\
{\bf Verification:} System test.
\end{reqlist}

\sreq{Location streams may be created}
\begin{reqlist}
{\bf Priority:} 1. \\
{\bf Source:} NCEP-SSI, PSAS, MIT, GFDL. \\
{\bf Status:} Approved-2. \\
{\bf Verification:} Unit test. \\
{\bf Notes:} 
\end{reqlist}

\sreq{Location streams may be destroyed}
\begin{reqlist}
{\bf Priority:} 1. \\
{\bf Source:} NCEP-SSI, PSAS, MIT, GFDL. \\
{\bf Status:} Approved-2. \\
{\bf Verification:} Unit test. \\
{\bf Notes:} 
\end{reqlist}

\sreq{Location streams may be copied}
Given an existing location stream, a new stream may be generated with a and
possibly a different length.
\begin{reqlist}
{\bf Priority:} 2. \\
{\bf Source:} PSAS, MIT, GFDL.\\
{\bf Status:} Approved-2. \\
{\bf Verification:} Unit test.
\end{reqlist}

\sreq{Reading streams}
Location streams may be read from files.
\begin{reqlist}
{\bf Priority:} 1. \\
{\bf Source:} Required by NCEP-SSI, MIT, GFDL. \\
{\bf Status:} Approved-1. \\
{\bf Verification:} Unit test. \\
{\bf Notes:} I no longer find this necessary. It is sufficient to have an IO requirement for fields (which can be defined on location streams) - Arlindo. - ditto CNH (but it is a requirement!).
\end{reqlist}

\sreq{Writing streams}
Location streams may be written to files.
\begin{reqlist}
{\bf Priority:} 1. \\
{\bf Source:} Required NCEP-SSI, MIT, GFDL. \\
{\bf Status:} Approved-1. \\
{\bf Verification:} Unit test. \\
{\bf Notes:} See previous - Arlindo.
\end{reqlist}

\sreq{Background grid}
A location stream may have associated with it an underlying physical grid, so that each
location stream element may be uniquely associated with a single grid cell. (A single
grid cell may contain multiple location stream elements.)
\begin{reqlist}
{\bf Priority:} 1. \\
{\bf Source:} Required by NCEP-SSI, PSAS, MIT, GFDL \\
{\bf Status:} Approved-1. \\
{\bf Verification:} Unit test. \\
{\bf Notes:} This is necessary for such operations as halo updates on a location
stream. A background grid might be required because its hard to see how to
decompose the locations tream otherwise.
\end{reqlist}

\sreq{Location stream attributes}

\ssreq{Fixed length location streams}
Location streams may be of fixed length, specified at the time of generation.
\begin{reqlist}
{\bf Priority:} 1. \\
{\bf Source:} Required by PSAS, MIT, GFDL.\\
{\bf Status:} Approved-1. \\
{\bf Verification:} Unit test. 
\end{reqlist}

\ssreq{Extensible length location streams}
Location streams may be of extensible length
with an initial length specified at the time of generation.
\begin{reqlist}
{\bf Priority:} \\
{\bf Source:} \\
{\bf Status:} Deferred \\
{\bf Verification:} Unit test.\\
{\bf Notes:} Are we sure we want to reject? Could they just be
not as optimised. Imagine a location stream that is representing Lagrangian
elements in a decomposed fluid simulation. It could be useful to
be able to add particles as needed. It could be useful to delete
from middle etc...
\end{reqlist}

\ssreq{Global attributes: location stream name}
Each location stream has a unique name by which it can be referred.
\begin{reqlist}
{\bf Priority:} \\
{\bf Source:} \\
{\bf Status:} Rejected. \\
{\bf Verification:} Unit test. \\
{\bf Notes:}
\end{reqlist}

\ssreq{Location stream registry}
Upon creation, the name and a pointer to each location stream shall be stored in a
registry.  A pointer to any location stream may be determined given its name.
\begin{reqlist}
{\bf Priority:} \\
{\bf Source:} \\
{\bf Status:} Rejected. \\
{\bf Verification:} Unit test. \\
{\bf Notes:} Not part of PhysGrid
\end{reqlist}

\ssreq{Global attributes: number of dimensions}
A location stream may be queried for the number of dimensions, which is
set at the time of creation of the stream.
\begin{reqlist}
{\bf Priority:} 2. \\
{\bf Source:} Required by PSAS, MIT, GFDL.\\
{\bf Status:} Approved-2. \\
{\bf Verification:} Code inspection.
{\bf Note:} REJECT because of Arlindo {Elements in stream have similar properties}
\end{reqlist}

\ssreq{Global attributes: dimension names}
Each dimension has a name, which may be set and queried.
\begin{reqlist}
{\bf Priority:} 2. \\
{\bf Source:} Required by PSAS, MIT, GFDL.\\
{\bf Status:} Approved-2. \\
{\bf Verification:} Unit test.
\end{reqlist}

\ssreq{Global attributes: dimension units}
A location stream contains the units of each dimension, which may be set and queried.
\begin{reqlist}
{\bf Priority:} 2. \\
{\bf Source:} Required by PSAS, MIT, GFDL.\\
{\bf Status:} Approved-2. \\
{\bf Verification:} Unit test.
\end{reqlist}

\ssreq{Global attributes: text or numeric attributes}

A location stream may have an arbitrary number of text or numeric attributes,
which may be added, set and queried.  Each attribute has a text name by which it
can be queried.  Also, a location stream can be queried for a list of all global
attribute names.

\begin{reqlist}
{\bf Priority:} 2. \\
{\bf Source:} Required by NCEP-SSI, MIT, GFDL. \\
{\bf Status:} Approved-2. \\
{\bf Verification:} Unit test. \\
{\bf Notes:} Not quite sure what this is. Again, since we can define fields on location streams, not sure we need this here - Arlindo - ditto CNH (but it is a requirement!).
\end{reqlist}

\ssreq{Global attributes: number of elements and number in use}
The number of elements in a location stream is available.  For a fixed length stream,
both the total number of elements and the number of elements before the last
active element location may be queried.
\begin{reqlist}
{\bf Priority:} 2. \\
{\bf Source:} Required by NCEP-SSI, PSAS, MIT, GFDL.\\
{\bf Status:} Approved-2. \\
{\bf Verification:} Unit test. 
\end{reqlist}

\ssreq{Global attributes: null element location value}
Each location stream may indicate whether a particular location is active.

\begin{reqlist}
{\bf Priority:} \\
{\bf Source:} \\
{\bf Status:} Rejected. \\
{\bf Verification:} Code inspection. \\
{\bf Notes:} This functionality may be obtained with user-defined element
attributes, but this use will not be explicitly supported by ESMF.
\end{reqlist}


\ssreq{Elements in stream have similar properties}
All elements in a location stream will have the same numbers of dimensions, use the
same physical coordinates, the same units, the same element attributes (attributes at
some locations may be missing).
\begin{reqlist}
{\bf Priority:} 2. \\
{\bf Source:} MIT. \\
{\bf Status:} Rejected. \\
{\bf Verification:} Code inspection. \\
{\bf Notes:} MAJOR OBJECTION. Where did this come from? I'd like to be able to represent the whole observation vector for a given synoptic time on a single location stream. While data have horizontal coordinates in (lat,lon), vertical coordinates may vary widely (winds at 10m above sfc, temperature at 500 hPa, to name a few). So, each element would contain the (lat,lon,lev,levunits); it is conceivable that at  some point one would need (lat,lon,lev,xunits,yunits,zunits) - Arlindo.
NO STATUS ASSIGNED WE NEED TO REVISIT THIS - CNH/RWH.
*****WE****NEED****TO****CONSULT****ARLINDO*****
\end{reqlist}

\ssreq{Elements include values of locations}
Methods shall be provided to set and query each element's location.
\begin{reqlist}
{\bf Priority:} 1\\
{\bf Source:} Required by PSAS, MIT, GFDL.\\
{\bf Status:} Approved-1. \\
{\bf Verification:} Unit test.
\end{reqlist}

\ssreq{Elements may be copied}
Methods shall be provided to copy all information from one element to another. 
When elements are copied from one location stream to another, all corresponding
properties and attributes will be copied, while missing attributes will be
set to missing or a user-provided default.
\begin{reqlist}
{\bf Priority:} 1\\
{\bf Source:} Required by PSAS, GFDL, MIT.\\
{\bf Status:} Approved-2. \\
{\bf Verification:} Unit test. 
\end{reqlist}

\ssreq{Elements may have attributes}
Text or data attributes may be attached to each element.  These
attributes may be null for any particular element.  Methods shall be provided to set
and query each element's attribute.  Element attributes may use standard names to
promote interoperability.
\begin{reqlist}
{\bf Priority:} 3\\
{\bf Source:} \\
{\bf Status:} Proposed. \\
{\bf Verification:} Unit test. \\
{\bf Notes:} I don't see the point of this as one can define fields on location streams - Arlindo.
\end{reqlist}

\ssreq{Location streams may contain null (discarded) elements}

Some elements within a location stream may be set to be invalid.  This may be a way
to specify the elements that are irrelevant for a particular subdomain.

\begin{reqlist}
{\bf Priority:} \\
{\bf Source:} \\
{\bf Status:} Rejected. \\
{\bf Verification:} Unit test. \\
{\bf Notes:} 
\end{reqlist}

\ssreq{Location streams may be queried for valid elements}

Location streams may be queried to obtain an ordered list of the indices of (or
pointers to) all valid elements.

\begin{reqlist}
{\bf Priority:} \\
{\bf Source:} \\
{\bf Status:} Rejected. \\
{\bf Verification:} Unit test. 
\end{reqlist}

%-------------------------------------------------------------------------------
\sreq{Location stream methods requiring registries of dependent data}
If all of the data streams that use a particular location stream are known,
additional methods for manipulating location streams and associated data streams are
possible.
\begin{reqlist}
{\bf Priority:} \\
{\bf Source:} \\
{\bf Status:} Rejected. \\
{\bf Verification:} Unit test. \\
{\bf Notes:} I don't know what this means - CNH.
\end{reqlist}

\ssreq{Registry of data streams}
Each location stream includes a registry of all the data streams that rely upon a
location stream.  This is necessary for location streams and data streams to be
manipulated in compatible ways.
\begin{reqlist}
{\bf Priority:}  \\
{\bf Source:} \\
{\bf Status:} Rejected. \\
{\bf Verification:} Unit test. \\
{\bf Notes:} See previous req (what is a "data stream") - CNH.
\end{reqlist}

\ssreq{Extensible location streams may be extended}
\begin{reqlist}
{\bf Priority:}  \\
{\bf Source:} \\
{\bf Status:} Rejected. \\
{\bf Verification:} Unit test. \\
{\bf Notes:} Who rejected extensible location streams? Who proposed them? - CMD
\end{reqlist}

\ssreq{Extensible location streams may be shortened}
\begin{reqlist}
{\bf Priority:}  \\
{\bf Source:} \\
{\bf Status:} Rejected. \\
{\bf Verification:} Unit test.
\end{reqlist}

\ssreq{Extensible length location streams may be converted to fixed length}
\begin{reqlist}
{\bf Priority:}  \\
{\bf Source:} \\
{\bf Status:} Rejected. \\
{\bf Verification:} Unit test. 
\end{reqlist}

\ssreq{Fixed length location streams may be converted to extensible length}
\begin{reqlist}
{\bf Priority:}  \\
{\bf Source:} \\
{\bf Status:} Rejected. \\
{\bf Verification:} Unit test.
\end{reqlist}

\ssreq{Fixed length streams may have null elements moved to end}
\begin{reqlist}
{\bf Priority:}  \\
{\bf Source:} \\
{\bf Status:} Rejected. \\
{\bf Verification:} Unit test. 
\end{reqlist}

%===============================================================================
\req{Physical grids}
%-------------------------------------------------------------------------------

A physical grid identifies a set of locations in physical space.  
Local physical grids provide the locations of each of the cells/points
associated with the range of indices in a distributed grid.  A local physical grid is
associated with a single distributed grid.  It may have undistributed dimensions that 
are not present in the underlying distributed grid. Multiple local 
physical grids may be derived from the same global, undistributed physical grid. 

Physical grids may be purely horizontal (one dimensional or two dimensional), purely vertical, or 3-dimensional. 
Structured grids assume that adjacent locations in index space share boundaries
in a predictable way.  Unstructured grids also have concepts of neighboring
cells, but the relative indices of neighbors are unpredictable.

\begin{reqlist}
{\bf Priority:} 1. \\
{\bf Source:} Required by CAM-EUL, CLM, CCSM-CPL, POP, CICE, NCEP-GSM, NCEP-SSI,
     CAM-FV, PSAS, MIT, WRF, GFDL. \\
{\bf Status:} Approved-1. \\
{\bf Verification:} System test.
\end{reqlist}

\sreq{Reading grids}
Given a distributed grid, a local physical grid can be read from a standard file containing a
global physical grid. If no distributed grid is provided, the global physical grid will
be read in.
\begin{reqlist}
{\bf Priority:} 1. \\
{\bf Source:} Required by CAM-EUL, CLM, CCSM-CPL, POP, CICE, 
              CAM-FV, PSAS, MIT, GFDL. \\
{\bf Status:} Approved-1. \\
{\bf Verification:} Unit test.
\end{reqlist}

\sreq{Writing grids}
Physical grids can be output to standard files.
\begin{reqlist}
{\bf Priority:} 1. \\
{\bf Source:} Required by CCSM-CPL, POP, CICE, 
              CAM-FV, PSAS, MIT, GFDL. \\
{\bf Status:} Approved-1. \\
{\bf Verification:} Unit test.\\
{\bf Notes:}  Here a standard file means that the file will
be suitable for holding global physical grid, independent of distribution.
\end{reqlist}

\sreq{Local physical grids may be internally generated}
For an arbitrary number of points in the global domain of the associated
distributed grid, it may be possible to specify an algorithm for internally
determining the local physical grid.
\begin{reqlist}
{\bf Priority:} 1. \\
{\bf Source:} Required by POP, CICE, NCEP-GSM, NCEP-SSI,
              CAM-FV, PSAS, MIT, WRF, GFDL. \\
{\bf Status:} Approved-1. \\
{\bf Verification:} Unit test.
\end{reqlist}

\sreq{Null physical grid creation}
It shall be possible to create any data objects associated with a physical grid without
providing the data that a physical grid will contain.
\begin{reqlist}
{\bf Priority:} 3\\
{\bf Source:} Required by CCSM-CPL, MIT. \\
{\bf Status:} Approved-2. \\
{\bf Verification:} Unit test. \\
{\bf Notes:} Could be useful for testing.
\end{reqlist}

\sreq{Physical grid query.}
Methods shall be provided to query a physical grid for all information it contains.
\begin{reqlist}
{\bf Priority:} 1\\
{\bf Source:} Required by CAM-EUL, CLM, CCSM-CPL, POP, CICE, NCEP-GSM, NCEP-SSI,
              CAM-FV, PSAS, MIT, WRF, GFDL. \\
{\bf Status:} Approved-1. \\
{\bf Verification:} Unit test.\\
{\bf Notes:} Obvious
\end{reqlist}

\sreq{Cell specification}
Physical grids shall specify both the locations of cell vertices, and the locations
of cell centers.
\begin{reqlist}
{\bf Priority:} 1. \\
{\bf Source:} Required by POP, CICE, Regrid, NCEP-GSM, MIT, WRF, GFDL. \\
{\bf Status:} Approved-1. \\
{\bf Verification:} Unit test.\\
{\bf Notes:} Many models mix finite difference and finite volume concepts.
\end{reqlist}

\sreq{Refinement}
A physical grid may be interpolated to generate a physical grid with an equivalent span
 at finer or coarser resolution.  Methods should be provided to accomplish such
interpolation via a simple interface that uses the Regrid facility.
\begin{reqlist}
{\bf Priority:} 2. \\
{\bf Source:} Required by MIT(handled by regrid), GFDL(handled by regrid) \\
{\bf Status:} Approved-1. \\
{\bf Verification:} Unit test.\\
{\bf Notes:} Necessary for runtime configurable resolution.  Also, note that 
there may be a Catch-22 here, as Regrid would naturally provide the facility
for Regridding, but Regrid will typically require the target physical grid for
creating the Regridding.  This requirement is also explicitly addressed within the
Regrid requirement document.

** NEED TO CHECK THE INDEX SPACE REGRIDDING *** \\

** NEED A SPCIFICATION FOR HOW TO MAP BETWEEN TWO COMPATIBLE GRIDS SO THAT \\
** REGRID CAN FIGURE OUT ALIGNMENT BETWEEN TWO SPHERICAL POLAR GRIDS OR \\
** BETWEEN CARTESIAN AND CYLINDRICAL. MAYBE THE NOTION OF A "COMMON POINT" \\
** THAT COULD REPRESENT THE SAME LOCATION IN SPACE. \\
\end{reqlist}

\sreq{Regeneration}
A new local physical grid may be generated for a given distributed grid from another 
physical grid.
The span of the source physical grid may be the same as or a superset of
the span of the target.
\begin{reqlist}
{\bf Priority:} 2. \\
{\bf Source:} MIT, GFDL. \\
{\bf Status:} Approved-2. \\
{\bf Verification:} Unit test.\\
{\bf Notes:} Necessary for support of transposes, or of moving nests.
\end{reqlist}

\sreq{Distributed grid reference}
A local physical grid may be queried for the distributed grid upon which it is based.
\begin{reqlist}
{\bf Priority:} 2. \\
{\bf Source:}  MIT, GFDL.\\
{\bf Status:} Approved-1. \\
{\bf Verification:} Unit test. 
\end{reqlist}

\sreq{Horizontal coordinate independent of vertical}
Horizontal physical grid locations can be assumed independent of the vertical coordinate.
The horizontal metrics, however, may be function of the vertical coordinate, as
in thick-shell spherical coordinates.
\begin{reqlist}
{\bf Priority:} 1. \\
{\bf Source:} Any Objections? \\
{\bf Status:} Approved-1. \\
{\bf Verification:} Code inspection.\\
{\bf Notes:} If this assumption can be made, it greatly simplifies implementation.
No widely used counterexamples are known. This is a negative
requirement, i.e. something we are not going to do, there are no feaatures
required to support this.
\end{reqlist}

\sreq{Vertical coordinate potentially dependent on horizontal.}
Vertical physical grid locations may be functions of the horizontal coordinates, or may be
independent of them.
\begin{reqlist}
{\bf Priority:} 1. \\
{\bf Source:} GFDL-MOM4 (required), NSIPP, POP, 
              CAM-FV, PSAS, MIT, GFDL \\
{\bf Status:} Approved-1. \\
{\bf Verification:} Code inspection.\\
{\bf Notes:} This is necessary to support, for example, partial cells in
Z-coordinate ocean models.
\end{reqlist}

\sreq{Dimension extension}
A new physical grid may be generated by adding a dimension to an existing 
physical grid.  The span of the source physical grid may be the same as or a superset of
the span of the target.  For a local physical grid, the new dimension will be independent 
of the underlying distributed grid, and both local physical grids share the same 
distributed grid.  The new dimension may be in any order with respect to existing dimensions.
\begin{reqlist}
{\bf Priority:} 2 \\
{\bf Source:} CCSM-CPL, GFDL, MIT \\
{\bf Status:} Approved-1 \\
{\bf Verification:} Unit test\\
{\bf Notes:} Valuable for separating generation of vertical and horizontal
coordinates.
\end{reqlist}

\sreq{Dimension reduction}
A new physical grid may be generated by removing a dimension from an existing physical grid.
For a local physical grid, if the dimension that is removed is one that is present 
in the original underlying distributed grid, an appropriately reduced distributed 
grid must also be provided.  Otherwise the new local physical
grid is based on the same distributed grid as the original physical grid.
\begin{reqlist}
{\bf Priority:} 2 \\
{\bf Source:} CCSM-CPL, GFDL, MIT \\
{\bf Status:} Approved-1 \\
{\bf Verification:} Unit test\\
{\bf Notes:} This and {\it Dimension extension} have been made
Approved-1 beacuse the feature could be hard to add later.
\end{reqlist}

\sreq{Arbitrary dimensional physical grids}
Physical grids may have an arbitrary number of dimensions. 
\begin{reqlist}
{\bf Priority:} \\
{\bf Source:} ? \\
{\bf Status:} Rejected.\\
{\bf Verification:} Unit test.\\
{\bf Notes:} If supported, this facility would dramatically complicate implementation,
without adding much functionality.
\end{reqlist}

\sreq{1- 2- or 3- dimensional local physical grids}
Local physical grids may have up to 3 dimensions, but must have at least as many dimensions as the
underlying distributed grid. 
\begin{reqlist}
{\bf Priority:}  1\\
{\bf Source:} Required by CAM-EUL, CLM, CCSM-CPL, POP, CICE, NCEP-GSM, NCEP-SSI,
              CAM-FV, PSAS, MIT, WRF, GFDL. \\
{\bf Status:} Approved-1. \\
{\bf Verification:} Unit test.\\
\end{reqlist}

\sreq{Index order}
A physical grid may use any index order (XYZ, XZY, etc.).  Methods shall be provided to
specify the order upon creation and to query the order of a physical grid.  It shall
also be possible (if not efficient) to extract physical grid information in any index
order.
\begin{reqlist}
{\bf Priority:}  1 \\
{\bf Source:} Please list required orders in Fortran notation. CCSM-CPL(XY), 
CAM-EUL (XYZ, XZY, ZXY), POP(XYZ), CICE(XY), NCEP(XYZ,XZY,YXZ,ZXY,ZYX), 
WRF(XYZ,XZY,YXZ,ZXY,ZYX),
MIT(XYZ, XZY, ZXY, YZX),
PSAS (XYZ,XZY), GFDL (XYZ, ZXY, XZY, YZX) \\
{\bf Status:} Approved-1. \\
{\bf Verification:} Unit test.\\
{\bf Notes:} Necessary for support of transposes.
\end{reqlist}

\sreq{Dimension reordering}
A new local physical grid may be generated with reordered dimensions from another local physical grid.
If the new dimension order is inconsistent with the original distributed grid, a new
consistent distributed grid must also be provided.  To be consistent, all dimensions present
in a distributed grid must have the same relative order in the local physical grid.  (i.e. if the
distributed grid uses XY, local physical grids using XYZ, ZXY, or XZY are all consistent, while one using
ZYX is not.)
\begin{reqlist}
{\bf Priority:} 2\\
{\bf Source:} GFDL, MIT\\
{\bf Status:} Approved-2. \\
{\bf Verification:} Unit test.\\
{\bf Notes:} Necessary for support of transposes. We assume left most fastest memory ordering here.
\end{reqlist}

\sreq{Location index determination}
A method shall be provided to return the cell index of a location.  An option shall
be provided to either create an exception for any location outside of the valid
range of the coordinate system, or to produce a gracefully treatable return value if
the location is (1) outside of the range of the local physical grid, or (2) outside of the
range of the global physical grid.  The index locations should be floating point numbers to
facilitate interpolation.
\begin{reqlist}
{\bf Priority:} 1\\
{\bf Source:} Required by CCSM-CPL, POP, Regrid, MIT, GFDL. \\
{\bf Status:} Approved-1. \\
{\bf Verification:} Unit test.
\end{reqlist}

\sreq{Index location determination}
A method shall be provided to return the physical locations from a physical grid of
floating point index coordinates.  Any index in the global physical grid may be
used, although there may be performance differences between points that are on
and off of the local physical grid.
\begin{reqlist}
{\bf Priority:} 1\\
{\bf Source:} Regrid (maybe - depending on implementation). \\
{\bf Status:} Approved-1. \\
{\bf Verification:} Unit test. \\
{\bf Notes:} ** CLARIFY WITH REGRID **
\end{reqlist}

\sreq{Horizontal physical grids}

\ssreq{Physical grids map projections}
Physical grids may be generated from a number of standard map projections, including
traditional and Mercator grids on a sphere, rotated latitude-longitude,
tripolar, and Gaussian cylindrical grids.  Additional requested grids include 
cubed-sphere, polar stereographic, and Lambert conformal projections.
\begin{reqlist}
{\bf Priority:} 1. \\
{\bf Source:} Required by POP, CICE, NCEP-GSM, NCEP-SSI,
PSAS, MIT, GFDL. \\
{\bf Status:} Approved-1. \\
{\bf Verification:} Unit test.\\
{\bf Notes:}  Perhaps some of these should be read in from a file, rather than
internally generated - CNH?  The underlying (background?) grid associated with a location stream may be in one of these map projections - Arlindo.
** REVIEW NOTE ** \\
** WE NEED TO SPECIFY AN ORDER AND PRIORITY IN WHICH THE SPECIFIC GRIDS WILL BE IMPLEMENTED ** \\
** ONCE THE BASIC FEATURE IS IN PLACE                                                       ** \\
\end{reqlist}

\ssreq{Unstretched cartesian internal generation}
A simple interface shall be provided to internally generate a uniform Cartesian
coordinate physical grid, given the lengths of the edges of a square domain.
\begin{reqlist}
{\bf Priority:} 1. \\
{\bf Source:} MIT, GFDL. \\
{\bf Status:} Approved-1. \\
{\bf Verification:} Unit test. 
\end{reqlist}

\ssreq{Latitude-longitude internal generation}
A simple interface shall be provided to internally generate a uniform (constant
grid-spacing in degrees) latitude-longitude physical grid, given the extent of the domain
in latitude and longitude.                                                           
\begin{reqlist}
{\bf Priority:} 1. \\
{\bf Source:} CCSM-CPL, CAM-FV, PSAS, MIT, GFDL. \\
{\bf Status:} Approved-1. \\
{\bf Verification:} Unit test.
\end{reqlist}

\ssreq{Stand-alone global physical grid generation examples}
Stand-alone software examples shall be provided to demonstrate the generation of a
global physical grid file on a stretched latitude-longitude grid, a rotated
latitude-longitude grid and a tripolar grid. 
\begin{reqlist}
{\bf Priority:} 2. \\
{\bf Source:} GFDL, NCAR, 
DAO, MIT. \\
{\bf Status:} Approved-2 \\
{\bf Verification:} Unit test.\\
{\bf Notes:} These are intended both for real use, and for use as patterns in the
creation of physical grid files for more complicated grids.  The above list may be
altered, extended or reduced following discussions.
\end{reqlist}

\ssreq{Supported topologies} Supported horizontal grid topologies will include
logically rectangular grids that are reentrant (periodic) in 0, 1, or 2 directions,
northern and southern tripolar (Murray 1996), sphere, icosahedral, and unstructured
grids.  Unstructured arrays of logically rectangular grids [for cubed-sphere (Rancic
et al. 1996), reduced grids, and arbitrary nesting] will also be supported.
\begin{reqlist}
{\bf Priority:} 1. \\
{\bf Source:} Required by POP, CICE, NCEP-GSM,
CAM-FV, PSAS, MIT, GFDL.  \\
{\bf Status:} Approved-1. \\
{\bf Verification:} Unit test.\\
{\bf Notes:}  Topologies are intrinsic to both distributed grids and physical grids.  Since
the topology information is so widely used in distributed grids, and since distributed grids
are used to initiate local physical grids, it is perhaps reasonable to make topology a
property of a distributed grid, which is then inherited and checked by a local physical grid.
\end{reqlist}

\ssreq{Local physical grid topology consistency checking}
A mechanism shall be provided to verify that the locations of the points in
a local physical grid are consistent with the topology of the underlying distributed grid.  An
exception or warning shall be generated in case of inconsistency.
\begin{reqlist}
{\bf Priority:} 1. \\
{\bf Source:} Required by CCSM-CPL, POP, CICE, 
CAM-FV, PSAS, MIT, GFDL.  \\
{\bf Status:} Approved-2. \\
{\bf Verification:} Unit test.
\end{reqlist}

\ssreq{Areas tile sphere}
It may be specified that grid areas should be calculated using algorithms that
guarantee that the grid exactly (algorithmically to within 1 part in $10^{12}$) tiles
the sphere (or a portion of it). 
\begin{reqlist}
{\bf Priority:} 2. \\
{\bf Source:} GFDL (required), Regrid, CAM-FV, PSAS, MIT.  \\
{\bf Status:} Approved-2. \\
{\bf Verification:} Unit test.\\
{\bf Notes:} Needed to permit exact conservation of fluxes between models.
\end{reqlist}

\ssreq{Staggered grids}
Staggered grids will be supported as a single physical grid.
\begin{reqlist}
{\bf Priority:} 1. \\
{\bf Source:} Required by POP, CICE, CAM-FV, PSAS, MIT, GFDL.  \\
{\bf Status:} Approved-1. \\
{\bf Verification:} Code inspection.\\
{\bf Notes:} Standard requirement of a staggered grid.
\end{reqlist}

\ssreq{Available subgrids}
For locally quadrilateral horizontal grids, information shall be available for
each of the 4 related subgrids.  That is if a t-cell is centered at a tracer
point,  cells centered on the east face, north face, and northeast corner of
the t-cell will also be provided in the case of a NorthEast underlying
distributed grid.
\begin{reqlist}
{\bf Priority:} 2. \\
{\bf Source:} GFDL, MIT (required), POP, CICE. \\
{\bf Status:} Approved-1. \\
{\bf Verification:} Code inspection.\\
{\bf Notes:} Standard requirement of a staggered grid.
\end{reqlist}

\ssreq{Extensible grid point representations}
It is not anticipated that all possible grids will be included in
ESMF. It must, therefore, be relatively straightforward to add
new grids to the framework and to share those grid "extensions"
amongst the framework community. For example it should be possible
to add an icosahedral grid.
\begin{reqlist}
{\bf Priority:} 2\\
{\bf Source:} Required by POP(future icosahedral), CICE(future), 
PSAS, MIT, GFDL.  \\
{\bf Status:} Approved-2. \\
{\bf Verification:} Code inspection. \\
{\bf Notes:} This is basic to the extensibility of ESMF.
\end{reqlist}


\sreq{Horizontal functional representations}
A spectral horizontal description may be used.  More generally, the horizontal
structure of information may be given by specifying functional decompositions.
\begin{reqlist}
{\bf Priority:} 1 \\
{\bf Source:} MIT, GFDL, NCEP-GSM.  \\
{\bf Status:} Approved-1. \\
{\bf Verification:} Unit test.
\end{reqlist}

\ssreq{Horizontal Fourier grids} 
Cartesian Fourier grids will be supported.  Associated with this grid are the
wavenumbers (in units of $m^{-1}$) of each of the elements on the grid.
\begin{reqlist}
{\bf Priority:} 3 \\
{\bf Source:} GFDL (desired)\\
{\bf Status:} Proposed. \\
{\bf Verification:} Unit test.\\
{\bf Note:} This could be implemented for very little cost on the way to full spherical polar
spectral representation. However, no milestone requires this.\\
\end{reqlist}

\ssreq{Horizontal spherical harmonics grids} 
Spherical harmonics grids will be supported.  Associated with this grid are the
wavenumbers (nondimensional m,n) of each of the elements on the grid.  At a
minimum, rhomboidal and triangular truncations will be supported.
\begin{reqlist}
{\bf Priority:} 1. \\
{\bf Source:} Required by CCSM-CPL, CAM-EUL, NCEP-GSM, NCEP-SSI, GFDL. \\
{\bf Status:} Approved-1. \\
{\bf Verification:} Unit test.
\end{reqlist}

\ssreq{Mixed physical and Fourier grids}
Mixed physical and Fourier grids will be supported. In particular, a grid on the
sphere that is latitude in one dimension and Fourier zonal wavenumber
(nondimensional m) in the other dimension will be supported.
\begin{reqlist}
{\bf Priority:} 1 \\
{\bf Source:} Required by NCEP-SSI, NCEP-GSM, GFDL. \\
{\bf Status:} Approved-1. \\
{\bf Verification:} Unit test.
\end{reqlist}

\ssreq{Extensible horizontal functional representations}
The physical grid design should not preclude the user from using alternative
functional horizontal representations, such as spectral elements.
\begin{reqlist}
{\bf Priority:} 3 \\
{\bf Source:} \\
{\bf Status:} Proposed. \\
{\bf Verification:} Code inspection. \\
{\bf Notes:} This is basic to the extensibility of ESMF.
\end{reqlist}

\sreq{Vertical functional representations}
\ssreq{Vertical user defined functions}
The physical grid design should not preclude the user from using 
functional vertical representations, such as EOFs, eigenfunctions,
and finite elements.  The vertical coordinate of such a grid might be a wavenumber
or a similar quantity.
(???THIS PART MAY BE MORE APPROPRIATE FOR Regrid...) Support will provided for
accepting a user supplied matrix or function that would transform the function into some
vertical physical space. Regridding would then be able to perform the transform, the
inverse transform, and the adjoint transform.
\begin{reqlist}
{\bf Priority:} 1. \\
{\bf Source:} NCEP-SSI. \\
{\bf Status:} Approved-1. \\
{\bf Verification:} Code inspection. \\
{\bf Notes:} DO WE REALLY NEED THIS AS P1, A-1.
\end{reqlist}

\sreq{Area overlap checking}
A method shall be provided to check that physical grid cells do not overlap. 
\begin{reqlist}
{\bf Priority:} 2. \\
{\bf Source:} Required by GFDL, CCSM-CPL, Regrid, MIT. \\
{\bf Status:} Approved-2. \\
{\bf Verification:} Unit test.\\
{\bf Notes:} Standard self-consistency test.
\end{reqlist}

%-------------------------------------------------------------------------------
\sreq{Physical grid attributes}

\ssreq{Physical grid name}
Each physical grid has a unique name by which it can be referred.  If no name is
specified, one will automatically be generated.
\begin{reqlist}
{\bf Priority:} 2 \\
{\bf Source:} Required by CCSM-CPL, MIT. \\
{\bf Status:} Approved-2. \\
{\bf Verification:} Unit test.
\end{reqlist}

\ssreq{Number of dimensions}
A physical grid may be queried for the number of dimensions, which is
set at the time of its creation.  Corresponding local and global physical grids have
the same number of dimensions.
\begin{reqlist}
{\bf Priority:} 1. \\
{\bf Source:} Required by CCSM-CPL, Regrid, PSAS, MIT, GFDL.  \\
{\bf Status:} Approved-1. \\
{\bf Verification:} Unit test.
\end{reqlist}

\ssreq{Dimension names}
Each dimension has a name, which may be set and queried.  If no name is specified
for a dimension, a name will be automatically generated.
\begin{reqlist}
{\bf Priority:} 1. \\
{\bf Source:} CCSM-CPL, PSAS, MIT, GFDL. \\
{\bf Status:} Approved-1. \\
{\bf Verification:} Unit test.
\end{reqlist}

\ssreq{Dimension lengths}
A physical grid may be queried for the local or global lengths of each of its dimensions.
\begin{reqlist}
{\bf Priority:} 1. \\
{\bf Source:} Required by CCSM-CPL, Regrid, PSAS, MIT, GFDL. \\
{\bf Status:} Approved-1. \\
{\bf Verification:} Unit test.
\end{reqlist}

\ssreq{Dimension attributes and units}
A physical grid contains the units of each dimension, which may be set and queried. 
Dimensions may also have additional named attributes.
\begin{reqlist}
{\bf Priority:} 1. \\
{\bf Source:} Required by CCSM-CPL, Regrid, PSAS, MIT, GFDL.  \\
{\bf Status:} Approved-1. \\
{\bf Verification:} Unit test. 
\end{reqlist}

\ssreq{Global attributes}
A physical grid may have an arbitrary number of text or numeric attributes,
which may be added, set and queried.  Each attribute has a text name by which it
can be queried.  Also, a physical grid can be queried for a list of all global
attribute names.

\begin{reqlist}
{\bf Priority:} 2. \\
{\bf Source:} Required by CCSM-CPL, PSAS, MIT, GFDL. \\
{\bf Status:} Approved-2. \\
{\bf Verification:} Unit test.
\end{reqlist}

%===============================================================================
\req{Grid metrics}
%-------------------------------------------------------------------------------
Grid metrics are all of the lengths (or partial derivatives of distances with
index number) and related quantities required to do a variety of calculations. 
All metrics are a function of the grid and must be static with time.  Metric-like
fields that vary with time (thicknesses in isopycnal/isentropic coordinates
or node locations in fully Lagrangian codes) are not handled by the physical grid.
\begin{reqlist}
{\bf Priority:} 1 \\
{\bf Source:} Required by POP, CICE, MIT, GFDL. \\
{\bf Status:} Approved-1 \\
{\bf Verification:} System test.
\end{reqlist}

\sreq{Calculation of metrics}
All metrics may be calculated from grid locations.
\begin{reqlist}
{\bf Priority:} 1. \\
{\bf Source:} Required by MIT, GFDL.\\
{\bf Status:} Approved-1. \\
{\bf Verification:} Unit test.
\end{reqlist}

\sreq{Reading metrics}
All metrics may be read from a standard grid file.
\begin{reqlist}
{\bf Priority:} 1. \\
{\bf Source:} Required by POP, CICE, MIT, GFDL. \\
{\bf Status:} Approved-1. \\
{\bf Verification:} Unit test. 
\end{reqlist}

\sreq{MKS metric units}
Metrics have units of m or $m^2$, or other appropriate MKS units.
\begin{reqlist}
{\bf Priority:} 1. \\
{\bf Source:} Standard MKS requirement? \\
{\bf Status:} Approved-1. \\
{\bf Verification:} Code inspection. \\
{\bf Notes:} Does MKS allow things like Pascals? or is that SI - CNH.
Some metrics may not have units i.e. scale factors with latitude etc... - CNH.
\end{reqlist}

\sreq{Available metrics}
All metrics are optional.  In a particular instance of a physical grid it shall be
possible to specify which metric terms are available.  Typically, available
metric information includes an extensive list of grid lengths, cell areas, and
the angle between logical and physical north.  Some functionality (e.g. certain
Regridding) may be limited if certain common metric terms are omitted.
\begin{reqlist}
{\bf Priority:} 1. \\
{\bf Source:} POP, CICE, MIT, GFDL. \\
{\bf Status:} Approved-1. \\
{\bf Verification:} Code inspection. \\
{\bf Notes:} All models require some subset of this information.
\end{reqlist}

\sreq{On-demand metrics}
In cases where one metric can be generated internally either from grid information
or from other metrics or from another physical grid, a method may be provided to create
that metric field only once it is clear that it will be needed.
\begin{reqlist}
{\bf Priority:} 3 \\
{\bf Source:} POP(desired), MIT, GFDL. \\
{\bf Status:} Approved-2. \\
{\bf Verification:} Code inspection. \\
{\bf Notes:} With extensive metric information, this may be necessary to save space.
\end{reqlist}

\sreq{Query by name}
It shall be possible to query for a reference to a metric field by name.
\begin{reqlist}
{\bf Priority:} 1 \\
{\bf Source:} MIT, GFDL \\
{\bf Status:} Approved-1. \\
{\bf Verification:} Unit test.
\end{reqlist}

\sreq{Standard metric naming convention}
A standard metric naming convention will be specified or established to facilitate
the widespread use of metric information.  Individual applications need not
follow this convention, but may not achieve full functionality without it.
\begin{reqlist}
{\bf Priority:} 2 \\
{\bf Source:} GFDL, MIT \\
{\bf Status:} Approved-2. \\
{\bf Verification:} Code inspection.
\end{reqlist}

\sreq{Dimensionality of metrics}
In cases where metric terms are independent of one or more dimensions, they may be
stored in arrays that omit those dimensions.
\begin{reqlist}
{\bf Priority:} 2 \\
{\bf Source:} GFDL/HIM, POP, CICE, MIT. \\
{\bf Status:} Approved-2. \\
{\bf Verification:} Unit test. \\
{\bf Notes:} This may be necessary for adequate cache/register performance.  This may
need to be done at compile time?
\end{reqlist}

\sreq{Available structured horizontal quadrilateral grid metrics}
Available metric information may include an extensive list of grid lengths, cell
areas, and the angle between logical and physical north.
\begin{reqlist}
{\bf Priority:} 1. \\
{\bf Source:} GFDL, POP, CICE, MIT. \\
{\bf Status:} Approved-1. \\
{\bf Verification:} Code inspection. \\
{\bf Notes:} All using quadrilateral horizontal models require some subset of this information.
\end{reqlist}

\ssreq{Cell areas}
Cell areas may be available for each of the 4 related subgrids.
\begin{reqlist}
{\bf Priority:} 1. \\
{\bf Source:} GFDL (required), POP(some), CICE(some), MIT. \\
{\bf Status:} Approved-1. \\
{\bf Verification:} Code inspection.
\end{reqlist}

\ssreq{Half-edge lengths}
Each of the 8 half-edge lengths may be available for each of the 4 related
subgrids.  Since neighboring cells share edges, it is desirable (although it violates
the proposed CF convention) to include only 4 fields.
\begin{reqlist}
{\bf Priority:} 2. \\
{\bf Source:} GFDL/MOM (required), MIT. \\
{\bf Status:} Approved-2. \\
{\bf Verification:} Code inspection.
\end{reqlist}

\ssreq{Center-to-edge distances}
Each of the 4 center to edge distances may be available for each of the 4 related
subgrids.
\begin{reqlist}
{\bf Priority:} 2. \\
{\bf Source:} GFDL/MOM (required), MIT. \\
{\bf Status:} Approved-2. \\
{\bf Verification:} Code inspection.
\end{reqlist}

\ssreq{Full-edge lengths}
Each of the 4 edge lengths may be available for each of the 4 related
subgrids.  Since neighboring cells share edges, it is desirable (although it violates
the proposed CF convention) to include only 2 fields.
\begin{reqlist}
{\bf Priority:} 2 \\
{\bf Source:} GFDL/HIM (required), POP, CICE, MIT \\
{\bf Status:} Approved-2 \\
{\bf Verification:} Code inspection.
\end{reqlist}

\ssreq{Edge-to-edge distances}
Both of the cell edge to edge distances may be available for each of the 4 related
subgrids.
\begin{reqlist}
{\bf Priority:} 2. \\
{\bf Source:} GFDL/HIM (required), MIT. \\
{\bf Status:} Approved-2. \\
{\bf Verification:} Code inspection. 
\end{reqlist}

\ssreq{Center-to-corner distances}
Each of the 4 center to corner distances may be available for each of the 4
related subgrids.
\begin{reqlist}
{\bf Priority:} 3. \\
{\bf Source:} MIT. \\
{\bf Status:} Proposed. \\
{\bf Verification:} Code inspection. \\
{\bf Notes:} This is used in some E-grid implementations.
\end{reqlist}

\ssreq{Cell orientation}
The angle at the cell center between logical and physical north may be available
for each of the 4 related subgrids.
\begin{reqlist}
{\bf Priority:} 2. \\
{\bf Source:} GFDL (required), POP, CICE, Regrid, MIT. \\
{\bf Status:} Approved-2. \\
{\bf Verification:} Code inspection. \\
{\bf Notes:} This is required for almost any non-latitude-longitude grid.
\end{reqlist}

\sreq{Available unstructured horizontal grid metrics}
Available metric information may include cell areas, edge lengths, and distances between
adjacent cell centers.
\begin{reqlist}
{\bf Priority:} 2. \\
{\bf Source:} Land Model? \\
{\bf Status:} Approved-2. \\
{\bf Verification:} Code inspection. \\
{\bf Notes:} All using unstructured horizontal grids require some subset of this
information.
\end{reqlist}

\sreq{Vertical metrics}
Available metric information may include spacing between cell centers and faces, in
units consistent with the vertical coordinate.
\begin{reqlist}
{\bf Priority:} 1. \\
{\bf Source:} Required by all. \\
{\bf Status:} Approved-2. \\
{\bf Verification:} Code inspection. \\
{\bf Notes:} All models require some subset of this information.
\end{reqlist}

\sreq{Cell volumes}
Available metric information may include 3-D cell volumes (or masses).  This is not
intended for use with models for which this quantity varies with time.
\begin{reqlist}
{\bf Priority:} 1. \\
{\bf Source:} GFDL/MOM, MIT (required), POP \\
{\bf Status:} Approved-2. \\
{\bf Verification:} Code inspection.
\end{reqlist}

\sreq{Methods for calculating metrics}
Metrics may be calculated by either standard or user-provided algorithms.
The following subrequirements provide a partial list of such algorithms,
which may be augmented later.
\begin{reqlist}
{\bf Priority:} 2. \\
{\bf Source:} Required by POP, CICE, MIT, GFDL. \\
{\bf Status:} Approved-2. \\
{\bf Verification:} Unit test. \\
{\bf Notes:} This is basic to the extensibility of ESMF. 
\end{reqlist}

\ssreq{Jacobian metric calculation}
Metrics may be calculated based on user-provided grid Jacobians [the matrix of
partial derivatives of the physical coordinates with respect to logical
coordinates (i.e. index space)], either in discrete form or as a series of function pointers.
\begin{reqlist}
{\bf Priority:} 3 \\
{\bf Source:} Regrid (desired). \\
{\bf Status:} Proposed. \\
{\bf Verification:} Unit test. 
\end{reqlist}

\ssreq{Spline metric calculation}
Metrics may be calculated by discrete estimates of the grid Jacobians based upon
the discrete grid locations.
\begin{reqlist}
{\bf Priority:} 2 \\
{\bf Source:} Regrid. \\
{\bf Status:} Approved-2. \\
{\bf Verification:} Unit test.
\end{reqlist}

\ssreq{Distance-based metric calculation}
Metrics may be calculated from distances (Great Circle on a sphere) between
the point locations of a physical grid.
\begin{reqlist}
{\bf Priority:} 3. \\
{\bf Source:} \\
{\bf Status:} Proposed. \\
{\bf Verification:} Unit test. \\
{\bf Notes:} This is basic to the extensibility of ESMF. 
\end{reqlist}

\sreq{Additional metrics}
It shall be possible for a user to specify additional metric terms to be associated
with a physical grid.
\begin{reqlist}
{\bf Priority:} 2. \\
{\bf Source:} CCSM-CPL, MIT, GFDL. \\
{\bf Status:} Approved-2. \\
{\bf Verification:} Unit test. \\
{\bf Notes:} This is basic to the extensibility of ESMF. 
\end{reqlist}


%===============================================================================
\req{Grid masks}
%-------------------------------------------------------------------------------

Grid masks are logical arrays on a grid that indicates whether the various
points on the grid are a part of a physically similar subdomain. For example,
masks are used to indicate which points are a part of the ocean and which are
land.  Masks are also important for nested applications.

\begin{reqlist}
{\bf Priority:} 1. \\
{\bf Source:} Required by POP, CICE, NCEP-GSM, NCEP-SSI, MIT, WRF, GFDL. \\
{\bf Status:} Approved-2. \\
{\bf Verification:} System test.
\end{reqlist}

\sreq{Arbitrary number of masks}
A physical grid may have an arbitrary number of masks associated with it.
\begin{reqlist}
{\bf Priority:} 2. \\
{\bf Source:} Required by POP, MIT, GFDL. \\
{\bf Status:} Approved-2. \\
{\bf Verification:} Code inspection.
\end{reqlist}

\sreq{Mask names}
Each of the masks associated with a physical grid is associated with a
unique name.  A method shall be specified to return a pointer to a mask given
its name.
\begin{reqlist}
{\bf Priority:} 2. \\
{\bf Source:} Required by MIT, GFDL. \\
{\bf Status:} Approved-2. \\
{\bf Verification:} Unit test.
\end{reqlist}

\sreq{Category masks}
Masks may have an arbitrary number of categories. (e.g. 1 for points in the
Atlantic, 2 for the Pacific, 3 for the Mediterranean, etc.)
\begin{reqlist}
{\bf Priority:} 2. \\
{\bf Source:} POP, CICE, MIT, GFDL. \\
{\bf Status:} Approved-2. \\
{\bf Verification:} Unit test.
\end{reqlist}

\sreq{Multiplicative masks}
Masks may consist of values between 0 and 1, for multiplicative masking.
\begin{reqlist}
{\bf Priority:} 2. \\
{\bf Source:} Required by POP, MIT, GFDL. \\
{\bf Status:} Approved-2. \\
{\bf Verification:} Unit test. 
\end{reqlist}

\sreq{Mask complement}
A method shall be provided to generate the complement of a mask.
\begin{reqlist}
{\bf Priority:} 3. \\
{\bf Source:} MIT, GFDL. \\
{\bf Status:} Proposed. \\
{\bf Verification:} Unit test.
\end{reqlist}


