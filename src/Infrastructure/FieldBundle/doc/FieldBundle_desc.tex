% $Id: FieldBundle_desc.tex,v 1.1.2.3 2008/07/17 21:03:52 feiliu Exp $

A FieldBundle functions mainly as a convenient container for storing
similar Fields.  It represents ``bundles'' of Fields that are 
discretized on the same Grid and distributed in the same manner.
It is an important data structure because this is often the form that
data being transferred between Components takes.  

Fields within a FieldBundle may be located at different locations relative 
to the vertices of their common Grid.  The Fields in a FieldBundle may
be of different dimensions, as long as the Grid dimensions that 
are distributed are the same.  For example, a surface Field on 
a distributed lat/lon Grid and a 3D Field with an added vertical 
dimension on the same distributed lat/lon Grid can be included
in the same FieldBundle.
 
FieldBundles can be created and destroyed, can have Attributes 
added or retrieved, and can have Fields added or retrieved.
Methods include queries that return information about the FieldBundle
itself and about the Fields that it contains.  The Fortran 
data pointer of a Field within a FieldBundle can be obtained 
by first retrieving the the Field with a call to {\tt ESMF\_FieldBundleGet},
and then using the {\tt ESMF\_FieldGet()} method to get the data.

FieldBundles can be added to States, which are used for sending
to or receiving data from Components.

In the future FieldBundles will serve as a mechanism for performance
optimization.  ESMF will take advantage of the similarities of the
Fields within a FieldBundle in order to implement collective communication,
IO, and regridding.  See Section \ref{sec:bundlerest} for a 
description of features that are being planned.





