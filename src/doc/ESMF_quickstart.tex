\section{Quick Start}
\label{sec:QuickStart}

This section gives a brief
description of to how to get the ESMF software, build it, 
and run the self-tests to verify the installation was successful.
More detailed information on each of these steps, as well as
information on running a {\tt quick\_start} application and linking
the ESMF with your own code is found in Sections \ref{sec:TechOver} 
and \ref{sec:TechOver2}.  

\subsection{Downloading ESMF}
\subsubsection{From the ESMF web site}
\label{sec:download}
ESMF is distributed as a source code tar file.  The tar file for the latest
public release, release notes, 
known bugs, supported platforms, documentation, and other related information 
can be found on the ESMF website, under the {\bf Download} tab:
\begin{verbatim}
    http://www.esmf.ucar.edu -> Download
\end{verbatim}
The source code for all other releases including the HEAD of the CVS trunk and the
last stable version can be found by following the {\bf View All Releases} link on the left hand navigation bar under {\bf Download}:
\begin{verbatim}
    http://www.esmf.ucar.edu -> Download -> View All Releases
\end{verbatim}

\subsubsection{From the SourceForge website}
ESMF can also be downloaded from the SourceForge website
from the {\bf Files} link on that website.
\begin{verbatim}
    http://sourceforge.net/projects/esmf -> Files
\end{verbatim}
Follow the directions on that web page to download a tar file.  

\subsection{Unpacking the download}
The source code comes as a zipped tar file. First unzip the file:
\begin{verbatim}
gunzip esmf*.tar.gz
\end{verbatim}

Then untar the file:
\begin{verbatim}
tar -xf esmf*.tar
\end{verbatim}

This will create a directory called {\tt esmf}.

\subsection{Directory Structure}
The current list of directories includes the following:
\begin{itemize}
\item README
\item application
\item build
\item build\_config
\item makefile
\item scripts
\item src
\end{itemize}

The build\_config directory contains subdirectories for
different operating system, compiler combinations. This is
a useful area to examine if porting ESMF to a new platform.

\subsection{Building ESMF}

After downloading and unpacking the ESMF tar file, the build procedure is:
\begin{enumerate}
\item Set the required environment variables. 
\item Type {\tt gmake info} to view and verifty your settings
\item Type {\tt gmake } to build the library.
\item Type {\tt gmake check } to run self-tests to verify
the build was successful.
\end{enumerate}
See the following sections for more information on each of these steps.

\subsubsection{Environment Variables}

The syntax for setting environment variables depends on which shell
you are running.  Examples of the two most common ways to set 
an environment variable are:
\begin{description}
\item[ksh] {\tt  export ESMF\_DIR=/home/joeuser/esmf}
\item[csh] {\tt  setenv ESMF\_DIR /home/joeuser/esmf}
\end{description}

The shell environment variables listed below are the ones most
frequently used.  There are others which address needs on specific
platforms or are needed under more unusual circumstances; 
see Section \ref{sec:TechOver} for the full list.  
\begin{description}

\item[ESMF\_DIR]
The environment variable {\tt ESMF\_DIR} must be set to the full pathname 
of the top level ESMF directory before building the framework.  This is the 
only environment variable which is required to be set on all platforms under 
all conditions.

\item[ESMF\_BOPT]
This environment variable controls the build option. To make a debuggable
version of the library set {\tt ESMF\_BOPT} to {\tt g} before building. The
default is {\tt O} (capital oh) which builds an optimized version of the 
library. If {\tt ESMF\_BOPT} is {\tt O}, {\tt ESMF\_OPTLEVEL} can also be set
to a numeric value between 0 and 4 to select a specific optimization level.

\item[ESMF\_COMM]
On systems with a vendor-supplied MPI communications library the vendor library 
is chosen by default for communications and {\tt ESMF\_COMM} need not be set.
For other systems (e.g. Linux or Darwin) a multitude of MPI implementations is
available and {\tt ESMF\_COMM} must be set to indicate which implementation is
used to build the ESMF library. Set {\tt ESMF\_COMM} according to your situation
to: {\tt mpich, mpich2, lam, openmpi} or {\tt intelmpi}. {\tt ESMF\_COMM} may
also be set to {\tt user} indicating that the user will set all the required
flags using advanced ESMF environment variables.

Alternatively, ESMF comes with a single-processor MPI-bypass library which is
the default for Linux and Darwin systems. To force the use of this bypass
library set {\tt ESMF\_COMM} equal to "mpiuni".

\item[ESMF\_COMPILER]
The ESMF library build requires a working Fortran90 and C++ compiler. On 
platforms that don't come with a single vendor supplied compiler suite
(e.g. Linux or Darwin) {\tt ESMF\_COMPILER} must be set to select which Fortran
and C++ compilers are being used to build the ESMF library. Notice that setting
the {\tt ESMF\_COMPILER} variable does {\em not} affect how the compiler
executables are located on the system. {\tt ESMF\_COMPILER} (together with
{\tt ESMF\_COMM}) affect the name that is expected for the compiler executables.
Furthermore, the {\tt ESMF\_COMPILER} setting is used to select compiler and
linker flags consistent with the compilers indicated.

By default Fortran and C++ compiler executables are expected to be located in
a location contained in the user's {\tt PATH} environment variable. This means
that if you cannot locate the correct compiler executable via the {\tt which}
command on the shell prompt the ESMF build system won't find it either!

There are advanced ESMF environment variables that can be used to select 
specific compiler executables by specifying the full path. This can be used to
pick specific compiler executables without having to modify the {\tt PATH}
environment variable.

Use 'gmake info' to see which compiler executables the ESMF build system will
be using according to your environment variable settings.

To see possible values for {\tt ESMF\_COMPILER}, cd to 
{\tt \$ESMF\_DIR/build\_config} and list the directories there. The first part 
of each directory name corresponds to the output of 'uname -s' for this 
platform. The second part contains possible values for {\tt ESMF\_COMPILER}. In
some cases multiple combinations of Fortran and C++ compilers are possible, e.g.
there is {\tt intel} and {\tt intelgcc} available for Linux. Setting 
{\tt ESMF\_COMPILER} to {\tt intel} indicates that both Intel Fortran and 
C++ compilers are used, whereas {\tt intelgcc} indicates that the Intel Fortran
compiler is used in combination with GCC's C++ compiler.

If you do not find a configuration that matches your situation you will need to
port ESMF.

\item[ESMF\_ABI]
If a system supports 32-bit and 64-bit (pointer wordsize) application binary
interfaces (ABIs), this variable can be set to select which ABI to use. Valid 
values are {\tt 32} or {\tt 64}. By default the most common ABI is chosen.

\item[ESMF\_SITE]
The SourceForge {\tt esmfcontrib} repository contains makefiles which have 
already been customized for certain machines.  If one exists for your site 
and you wish to use it, download the corresponding files into the 
{\tt build\_contrib} directory and set {\tt ESMF\_SITE} to your location
(which corresponds to the last part of the directory name).  See the 
SourceForge site {\tt http://sourceforge.net/projects/esmfcontrib} for more 
information.

\item[ESMF\_INSTALL\_PREFIX]
This variable specifies the prefix of the installation path used during the
installation process accessible thought the install target. Libraries, F90
module files, header files and documentation all are installed relative to
{\tt ESMF\_INSTALL\_PREFIX} by default. The {\tt ESMF\_INSTALL\_PREFIX} may be
provided as absolute path or relative to {\tt ESMF\_DIR}.

\end{description}


\subsubsection{GNU make}
The ESMF build system uses the GNU make program; it is normally named 
{\tt gmake} but may also be simply {\tt make} or {\tt gnumake} on some 
platforms (we will use {\tt gmake} in this document). ESMF does not use
configure or autoconf;  the selection of various options is done by
setting environment variables before building the framework. 


\subsubsection{gmake info}
{\tt gmake info} is a command that assists the user in verifying that the ESMF 
variables have been set appropriatly. It also tells the user the paths to 
various libraries e.g. MPI that are set on the system. The user to review 
this information to verify their settings. In the case of a build failure, 
this information is invaluable and will be the first thing asked for by the
ESMF support team. Below is an {/b example output} from {/tt gmake info}: 
 
\begin{verbatim}

--------------------------------------------------------------
Make version:
GNU Make version 3.79, by Richard Stallman and Roland McGrath.
Built for powerpc-apple-darwin7.0
Copyright (C) 1988, 89, 90, 91, 92, 93, 94, 95, 96, 97, 98, 99
	Free Software Foundation, Inc.
This is free software; see the source for copying conditions.
There is NO warranty; not even for MERCHANTABILITY or FITNESS FOR A
PARTICULAR PURPOSE.

Report bugs to <bug-make@gnu.org>.


--------------------------------------------------------------
Fortran Compiler version:
ERROR: No input files.
Pro Fortran 9.0

--------------------------------------------------------------
C++ Compiler version:
g++ (GCC) 3.3 20030304 (Apple Computer, Inc. build 1666)
Copyright (C) 2002 Free Software Foundation, Inc.
This is free software; see the source for copying conditions.  There is NO
warranty; not even for MERCHANTABILITY or FITNESS FOR A PARTICULAR PURPOSE.


--------------------------------------------------------------
Preprocessor version:
gcc (GCC) 3.3 20030304 (Apple Computer, Inc. build 1666)
Copyright (C) 2002 Free Software Foundation, Inc.
This is free software; see the source for copying conditions.  There is NO
warranty; not even for MERCHANTABILITY or FITNESS FOR A PARTICULAR PURPOSE.


--------------------------------------------------------------
ESMF_VERSION_STRING "3.1.0r beta snapshot"
--------------------------------------------------------------
 
--------------------------------------------------------------
 * User set ESMF environment variables *
ESMF_OS=Darwin
ESMF_DIR=/Users/svasquez/script_dirs/branch_builds/esmf
ESMF_TESTWITHTHREADS=OFF
ESMF_INSTALL_PREFIX=/Users/svasquez/script_dirs/branch_builds/esmf/../install_dir
ESMF_COMM=mpiuni
ESMF_TESTEXHAUSTIVE=ON
ESMF_SITE=default
ESMF_ABI=32
ESMF_COMPILER=absoft
 
--------------------------------------------------------------
 * ESMF environment variables *
ESMF_DIR: /Users/svasquez/script_dirs/branch_builds/esmf
ESMF_OS:                Darwin
ESMF_MACHINE:           Power Macintosh
ESMF_ABI:               32
ESMF_COMPILER:          absoft
ESMF_BOPT:              g
ESMF_COMM:              mpiuni
ESMF_SITE:              default
ESMF_PTHREADS:          ON
ESMF_ARRAY_LITE:        FALSE
ESMF_NO_INTEGER_1_BYTE: FALSE
ESMF_NO_INTEGER_2_BYTE: FALSE
ESMF_FORTRANSYMBOLS:    default
ESMF_TESTEXHAUSTIVE:    ON
ESMF_TESTWITHTHREADS:   OFF
ESMF_TESTMPMD:          OFF
 
--------------------------------------------------------------
 * ESMF environment variables pointing to 3rd party software *
 
--------------------------------------------------------------
 * ESMF environment variables for final installation *
ESMF_INSTALL_PREFIX:    /Users/svasquez/script_dirs/branch_builds/esmf/../install_dir
ESMF_INSTALL_HEADERDIR: include
ESMF_INSTALL_MODDIR:    mod/modg/Darwin.absoft.32.mpiuni.default
ESMF_INSTALL_LIBDIR:    lib/libg/Darwin.absoft.32.mpiuni.default
ESMF_INSTALL_DOCDIR:    doc
 
 
--------------------------------------------------------------
 * Compilers, Linkers, Flags, and Libraries *
Location of the preprocessor:      /usr/bin/gcc
Location of the Fortran compiler:  /Applications/Absoft/bin/f90
Location of the Fortran linker:    /Applications/Absoft/bin/f90
Location of the C++ compiler:      /usr/bin/g++
Location of the C++ linker:        /usr/bin/g++

Fortran compiler flags:
ESMF_F90COMPILEOPTS: -g -YEXT_NAMES=LCS -YEXT_SFX=_
ESMF_F90COMPILEPATHS: -p/Users/svasquez/script_dirs/branch_builds/esmf/mod/modg/Darwin.absoft.32.mpiuni.default -I/Users/svasquez/script_dirs/branch_builds/esmf/src/include
ESMF_F90COMPILECPPFLAGS:  -DESMF_TESTEXHAUSTIVE -DS32=1 -DESMF_OS_Darwin=1 -DESMF_MPIUNI

Fortran linker flags:
ESMF_F90LINKOPTS:  -YEXT_NAMES=LCS -YEXT_SFX=_
ESMF_F90LINKPATHS: -L/Users/svasquez/script_dirs/branch_builds/esmf/lib/libg/Darwin.absoft.32.mpiuni.default
ESMF_F90LINKRPATHS: 
ESMF_F90LINKLIBS:  -lstdc++
ESMF_F90ESMFLINKLIBS: -lesmf  -lstdc++

C++ compiler flags:
ESMF_CXXCOMPILEOPTS: -g -DESMF_LOWERCASE_SINGLEUNDERSCORE
ESMF_CXXCOMPILEPATHS: -I/Users/svasquez/script_dirs/branch_builds/esmf/src/include -I/Users/svasquez/script_dirs/branch_builds/esmf/src/Infrastructure/stubs/mpiuni
ESMF_CXXCOMPILECPPFLAGS: -DESMF_TESTEXHAUSTIVE  -DS32=1 -DESMF_OS_Darwin=1 -D__SDIR__='' -DESMF_MPIUNI

C++ linker flags:
ESMF_CXXLINKOPTS: 
ESMF_CXXLINKPATHS: -L/Users/svasquez/script_dirs/branch_builds/esmf/lib/libg/Darwin.absoft.32.mpiuni.default -L/Applications/Absoft/lib
ESMF_CXXLINKRPATHS: 
ESMF_CXXLINKLIBS:  -lf90math -lfio -lac -lf77math -lm
ESMF_CXXESMFLINKLIBS: -lesmf  -lf90math -lfio -lac -lf77math -lm


--------------------------------------------------------------
Compiling on Mon Apr 7 03:11:21 MDT 2008 on glass.image.ucar.edu
Machine characteristics: Darwin glass.image.ucar.edu 7.9.0 Darwin Kernel Version 7.9.0: Wed Mar 30 20:11:17 PST 2005; root:xnu/xnu-517.12.7.obj~1/RELEASE_PPC Power Macintosh powerpc
==============================================================

\end{verbatim}


\subsubsection{Building Makefile Targets}

The makefiles follow the GNU target standards where possible.
The most frequently used targets for building are listed below:
\begin{description}
\item[lib] build the ESMF libraries only (default)
%\item[all] build the libraries, unit and system tests, examples, and demos
\item[all] build the libraries, unit and system tests and examples
\item[doc] build the documentation (requires specific latex macros packages
and additional utilities; see Section \ref{sec:TechOver} for more details
on the requirements).  
\item[info] print out extensive system configuration information about what
           compilers, libraries, paths, flags, etc are being used
\item[clean] remove all files built for this platform/compiler/wordsize.
\item[clobber] remove all files built for all architectures
\item[install] install the ESMF library in a custom location
\end{description}


\subsubsection{Testing Makefile Targets}

To build and run the unit and system tests in non-exhaustive mode, type:
\begin{verbatim}
gmake check
\end{verbatim}
A summary report of success and failures will be printed out at the end.

\noindent Other test-related targets are:
\begin{description}
%\item[all\_tests] build and run all available tests and demos
\item[all\_tests] build and run all available tests
\item[build\_all\_tests] build tests only, do not execute
\item[run\_all\_tests] run tests without rebuilding and print a
summary of the results
\item[check\_all\_tests] 
print out the results summary without re-executing the tests again
%\item[clean\_all\_tests] remove all test and demo executables 
\item[clean\_all\_tests] remove all test executables 
\end{description}

For all the targets listed above, the string {\tt all\_tests} can be
replaced with one of the strings listed below to select a
specific type of test:
\begin{description}
\item[unit\_tests] unit tests exercise a single part of the system
\item[system\_tests] system tests combine functions across the system
\item[examples] examples contain code illustrating a single type of function
%\item[demos] demos are example applications showing the use of the system
\end{description}
For example, {\tt gmake build\_examples} recompiles the example programs but 
does not execute them.  {\tt gmake clean\_system\_tests} removes all
executables and files associated with the system tests.

For the unit tests only, there is an additional environment variable
which affects how the tests are built:
\begin{description}
\item[ESMF\_TESTEXHAUSTIVE]
If this variable is set to {\tt ON} before compiling the unit tests,
longer and more exhaustive unit tests will be run.  Note that this is a
compile-time and not run-time option.
\end{description}

