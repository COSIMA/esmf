

\section{The ESMF Reference Manual for Fortran}

ESMF provides a complete set of Fortran interfaces and
some C and C++ interfaces.  This {\it ESMF Reference Manual} is a listing of 
ESMF standard interfaces for Fortran.\footnote{Since the audience for it is 
small, we have not yet prepared a comprehensive reference manual for C 
or C++.}  

Interfaces are grouped by class.  A class is an object-oriented software 
design construct that embodies 
a specific concept like a physical field.  Superstructure classes 
are listed first in this {\it Manual}, followed by infrastructure 
classes.

The major classes in the ESMF superstructure are Components, which 
typically represent
large pieces of functionality such as models, model couplers, and 
dynamics and physics packages; and States, which are the data structures
used to communicate data between Components.  There are both data
structures and utilities in the ESMF 
infrastructure; classes include Fields, collections of Fields on the 
same grid (called FieldBundles), Arrays, and utilities for communication,
decomposition, time management, and application configuration.

\begin{center}
\begin{figure}
\caption{Schematic of the ESMF ``sandwich'' architecture. In this
design the framework consists of two parts, an upper level
{\bf superstructure} layer and a lower-level {\bf infrastructure} layer.
User code is sandwiched between these two layers.}
\label{fig:TheESMFwich}
\scalebox{1.0}{\includegraphics{ESMF_sandwich}}
\end{figure}
\end{center}

