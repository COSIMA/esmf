\section{Conventions}
\label{sec:conventions}

\subsection{Typeface and Diagram Conventions}

\input{ESMF_typefaceconventions}

\subsection{Method Name and Argument Conventions}

Method names begin with {\tt ESMF\_}, followed by the class name, 
followed by the name of the operation being performed.  Each new 
word is capitalized.  Although Fortran interfaces are not case-sensitive,
we use case to help parse multi-word names.  

For method arguments that are multi-word, the first word is lower
case and subsequent words begin with upper case.  ESMF class 
names (including typed flags) are an exception.  When multi-word 
class names appear in argument lists, all letters after the first 
are lower case.  The first letter is lower case if the class is the
first word in the argument and upper case otherwise.  For 
example, in an argument list the DELayout class name may appear 
as {\tt delayout} or {\tt srcDelayout}.

Most Fortran calls in the ESMF are subroutines, with 
any returned values passed through the interface.  For the sake of 
convenience, some ESMF calls are written as functions.

A typical ESMF call looks like this:

\begin{verbatim}
call ESMF_<ClassName><Operation>(classname, firstArgument, 
           secondArgument, ..., rc)
\end{verbatim}

where \newline
{\tt <ClassName>} is the class name, \newline
{\tt <Operation>} is the name of the action to be performed, \newline 
{\tt classname} is a variable of the derived type associated 
with the class, \newline
the {\tt arg*} arguments are whatever other variables are required 
for the operation, \newline
and {\tt rc} is a return code. \newline

\subsection{Locating Methods in this Manual}

Methods for each class are located in the section devoted to 
that class in the {\it Reference Manual}.  In some classes, method
listings
are split into a number of different sections.  For example, there may
be separate listings for basic Field methods and overloaded Field
methods.  The methods in each listing are ordered alphabetically.
The split into different
listings is a side effect of the automated document generation system
we use; it reflects which methods are located in the same source files.
Eventually we hope to eliminate this issue and have all methods in
a class listed together.


