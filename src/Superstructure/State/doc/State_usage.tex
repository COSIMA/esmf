% $Id: State_usage.tex,v 1.10.2.2 2008/04/05 03:14:23 cdeluca Exp $
%
% Earth System Modeling Framework
% Copyright 2002-2008, University Corporation for Atmospheric Research, 
% Massachusetts Institute of Technology, Geophysical Fluid Dynamics 
% Laboratory, University of Michigan, National Centers for Environmental 
% Prediction, Los Alamos National Laboratory, Argonne National Laboratory, 
% NASA Goddard Space Flight Center.
% Licensed under the University of Illinois-NCSA License.

%\subsection{Use and Examples}

A Gridded Component generally has one associated import 
State and one export State.  Generally the States 
associated with a Gridded Component will be created by 
the Gridded Component's parent component.
In many cases, the States will be created containing 
no data.  Both the empty States and the
newly created Gridded Component are passed
by the parent component into the Gridded Component's initialize 
method.  This is where the States get prepared for use 
and the import State is first filled with data.

States can be created without the Fields, Arrays, FieldBundles,
and other States they will eventually contain in a number 
of ways.  They can be created with names as placeholders where 
these data items will eventually be.  When the States are passed 
into the Gridded Component's initialize method, Field,
FieldBundle, and Array create calls can be made in that method
to replace the name placeholders with real data objects.

States can also be filled with data items that do not yet 
have data allocated.  Fields, FieldBundles, and Arrays each have 
methods that support their creation without actual data 
allocation - the grid and metadata are set up but no
Fortran array of data values is allocated.  In this approach, 
when a State is passed into its associated Gridded Component's 
initialize method, the incomplete Arrays, Fields, and 
FieldBundles within the State can allocate or reference data 
inside the initialize method.

States are passed through the interfaces of the Gridded 
and Coupler Components' run methods in order to carry data 
between the components.  While we expect
a Gridded Component's import State to be filled with data 
during initialization, its export State will typically be
filled over the course of its run method.  At the end of
a Gridded Component's run method, the filled export State 
is passed out through the argument list into a Coupler 
Component's run method.  We recommend the convention that 
it enters the Coupler Component as the Coupler Component's
import State.  Here is it transformed into a form
that another Gridded Component requires, and passed out
of the Coupler Component as its export State.  It can then
be passed into the run method of a recipient Gridded Component
as that component's import State.

While the above sounds complicated, the rule is simple:
a State going into a component is an import State, and a 
State leaving a component is an export State.

Data items within a State can be marked needed or not needed,
depending on whether they are required for a particular 
application configuration.  If the item is marked not needed,
the user can make the Gridded Component's initialize method 
clever enough to not allocate the data for that item at all
and not compute it within the Gridded Component code.
For example, some diagnostics may not be desired for all 
runs.

Other flags will eventually be available for data items within 
a State, such as data ready for reading or writing, data valid or 
invalid, and data required for restart or not.  These are not
yet fully implemented, so only the default value for each value 
can be set at this time.

Objects inside States are normally created in {\tt unison} where
each PET executing a component makes the same object create call.
If the object contains data, like a Field, each PET may have a
different local chunk of the entire dataset but each Field has
the same name and is logically one part of a single distributed 
object.   As States are passed between components if any object
in a State was not created in unison on all the current PETs 
then some PETs have no object to pass into a
communication method (e.g. regrid or data redistribution).
A State method called reconcile must be called to broadcast 
information about these objects to all PETs in a component;
after which all PETs have a single uniform view of all objects.  

If components are running in sequential mode on all available PETs
and States are being passed between them there is no need to make
the reconcile call since all PETs have a uniform view of the objects.
However, if components are running on a subset of the PETs, as is
usually the case when running in concurrent mode, then when States
are passed into components which contain a superset of those PETs,
for example, a Coupler Component, all PETs must call reconcile
on the States before using them in any ESMF communication methods.
The reconcile process broadcasts metadata information about objects
which exist only on a subset of the PETs.  On PETs missing those
objects it creates a {\it proxy} object which contains any
attributes of the original object plus enough information for it
to be a data source or destination for a regrid or data redistribution
operation. 


